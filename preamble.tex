\usepackage[T1]{fontenc}
\usepackage[utf8]{inputenc}

\usepackage{kpfonts}

\usepackage[final]{graphicx}
\graphicspath{{./figs/}}

%% remove colorlinks when preparing for print
\usepackage[final,hyperindex,hyperfootnotes,bookmarksnumbered,colorlinks]{hyperref}

\usepackage{afterpage}

\newsubfloat{figure}

\usepackage{color}    % to color background of generated values values
\usepackage{seqsplit} % used in the output automatically generated by scripts
\usepackage{url}
\usepackage[textsize=footnotesize]{todonotes}
%% new command for box about missing references
\newcommand{\addref}[1][]{\todo[color=red!40,size=\tiny]{Add reference: #1}}

\usepackage[sectionbib,square,comma,numbers]{natbib}
\bibliographystyle{agu}

\usepackage{etoolbox}
\usepackage{stringstrings}
\usepackage{fp}
\usepackage{intcalc}
\usepackage{capt-of}

\usepackage{siunitx}
\DeclareSIUnit{\bp}{bp} % base pairs

\usepackage[english]{isodate}

\newfloat{supplement}{sup}{Supplementary}
\setfloatlocations{supplement}{h}
\renewcommand{\thesupplement}{S\arabic{supplement}}

%% Simple command for horizontal centering a single table cell
\newcommand{\centercell}[1]{\multicolumn{1}{c}{#1}}

\input{results/variables-configuration}
\newcommand{\CodingGenesInHistOne}{50}
\newcommand{\PseudoGenesInHistOne}{9}
\newcommand{\TotalGenesInHistOne}{59}

\newcommand{\CodingGenesInHistTwo}{11}
\newcommand{\PseudoGenesInHistTwo}{7}
\newcommand{\TotalGenesInHistTwo}{XXXXX}

\newcommand{\CodingGenesInHistThree}{3}
\newcommand{\PseudoGenesInHistThree}{XXXX}
\newcommand{\TotalGenesInHistThree}{XXXXX}

\newcommand{\CodingGenesInHistFour}{1}
\newcommand{\PseudoGenesInHistFour}{XXXX}
\newcommand{\TotalGenesInHistFour}{XXXXX}

\newcommand{\HistOneSpan}{2.3\,Mbp}
\newcommand{\HistTwoSpan}{105\,kbp}
\newcommand{\HistThreeSpan}{XXXXbp}
\newcommand{\HistFourSpan}{XXXXbp}

\input{results/variables-protein_stats}
\input{results/variables-utr}
\input{results/variables-histone_counts}
\input{results/variables-align_proteins_stats}
\input{results/variables-align_transcripts_stats}

%% We define a style here which will be used around all the values that
%% are copmuted automatically.  The idea is to place them in a grey box
%% so that readers can really get the idea that the values in the text
%% are automatically generated.  To do that, the scripts will put the
%% values inside a ScriptValue macro which we define here to control how
%% it really looks.
%%
%% By mixing a \strut with \fboxsep set to 0pt, we get an expanded
%% colorbox (not too tight), without changing the line height.  It
%% also solves the issue of having different colorbox with different
%% heights (see https://tex.stackexchange.com/questions/7530/height-of-colorbox )
\newcommand{\ScriptValue}[1]{\setlength{\fboxsep}{0pt}\colorbox[gray]{0.8}{\strut #1}}

%% But then, we also want to make operations with those values in LaTeX
%% using FPeval but FPeval fails because ScriptValue gets expanded into
%% something non-numeric.  So we use the following trick: we store the
%% original ScriptValue and FPeval macros, and then replace FPeval with
%% something that temporarily disables ScriptValue while we call the
%% original FPeval.
%% See https://tex.stackexchange.com/questions/159155/identify-pieces-of-text-automatically-generated-from-input-and-new-command
%% and https://tex.stackexchange.com/questions/283655/overloading-functions-of-the-fp-package
\let\RealScriptValue\ScriptValue

\let\RealFPeval\FPeval
\renewcommand{\FPeval}[2]{%
  \renewcommand{\ScriptValue}[1]{##1}%
  \RealFPeval{\UnmarkedResult}{#2}%
  \edef#1{\noexpand\ScriptValue{\UnmarkedResult}}%
  \renewcommand{\ScriptValue}{\RealScriptValue}%
}

\let\RealFPmin\FPmin
\renewcommand{\FPmin}[3]{%
  \renewcommand{\ScriptValue}[1]{##1}%
  \RealFPmin{\UnmarkedResult}{#2}{#3}%
  \edef#1{\noexpand\ScriptValue{\UnmarkedResult}}%
  \renewcommand{\ScriptValue}{\RealScriptValue}%
}

\let\RealFPround\FPround
\renewcommand{\FPround}[3]{%
  \renewcommand{\ScriptValue}[1]{##1}%
  \RealFPround{\UnmarkedResult}{#2}{#3}%
  \edef#1{\noexpand\ScriptValue{\UnmarkedResult}}%
  \renewcommand{\ScriptValue}{\RealScriptValue}%
}

\let\Realprintdate\printdate
\renewcommand{\printdate}[1]{%
  \renewcommand{\ScriptValue}[1]{##1}%
  \RealScriptValue{\Realprintdate{#1}}%
  \renewcommand{\ScriptValue}{\RealScriptValue}%
}

%% When using \SI, sometimes we are doing with "normal" values, othertimes
%% they are \ScriptValue macros.  In those cases, we will want to keep
%% ScriptValue box around the values, but we need to be careful to not
%% "promote" if not.  So we check of #1 is a macro (and assume it's
%% \ScriptValue if true.

\let\RealSI\SI
\renewcommand{\SI}[2]{%
  \ifdefmacro{#1}%
    {%
      \renewcommand{\ScriptValue}[1]{##1}%
      \RealScriptValue{\RealSI{#1}{#2}}%
      \renewcommand{\ScriptValue}{\RealScriptValue}%
    }%
    {%
      \RealSI{#1}{#2}%
    }%
}

\let\RealSIrange\SIrange
\renewcommand{\SIrange}[3]{%
  \ifdefmacro{#1}%
    {%
      \renewcommand{\ScriptValue}[1]{##1}%
      \RealScriptValue{\RealSIrange{#1}{#2}{#3}}%
      \renewcommand{\ScriptValue}{\RealScriptValue}%
    }%
    {%
      \RealSIrange{#1}{#2}{#3}%
    }%
}


%% A function to automatically round a number using the "best" SI prefix.
%% The idea is to transform very long numbers into something more
%% reasonable.  For example, instead of 1672945bp, say 1.67Mbp.
%% The first argument is the number, the second the length of the
%% decimal part, adn the third a SI unit (as defined by the siunitx package)
%% Examples:
%%  \AutoSIPrefix{12345}{2}{\bp} => \SI{12.35}{\kilo\bp}  % note rounding of 12.345
%%  \AutoSIPrefix{123456}{1}{\bp} => \SI{123.5}{\kilo\bp}
%%  \AutoSIPrefix{1234567}{0}{\bp} => \SI{1}{\mega\bp}
%%
%% As with the overloading of the FP* macros, we need to redefine \ScriptValue
\newcommand{\AutoSIPrefix}[3]{%
  \renewcommand{\ScriptValue}[1]{##1}%
  \stringlength[e]{#1}%
  \numdef{\length}{\theresult}%
%
  \numdef{\modulo}{\intcalcMod{\length}{3}}%
  \numdef{\power}{\intcalcMul{\intcalcDiv{\length}{3}}{3}}%
%% If modulo is zero, then we use the prefix below
  \ifnumequal{\modulo}{0}{\numdef{\modulo}{3} \numdef{\power}{\intcalcSub{\power}{3}}}{}%
%
  \ifnumequal{\power}{24}{\def\prefix{\yotta}}{%
    \ifnumequal{\power}{21}{\def\prefix{\zetta}}{%
      \ifnumequal{\power}{18}{\def\prefix{\exa}}{%
        \ifnumequal{\power}{15}{\def\prefix{\peta}}{%
          \ifnumequal{\power}{12}{\def\prefix{\tera}}{%
            \ifnumequal{\power}{9}{\def\prefix{\giga}}{%
              \ifnumequal{\power}{6}{\def\prefix{\mega}}{%
                \ifnumequal{\power}{3}{\def\prefix{\kilo}}{%
                  \def\prefix{}}}}}}}}}%
%% Compute the integer part
  \substring[q]{#1}{1}{\modulo}%
  \csedef{integer}{\thestring}%
%% Compute the fractional part
  \numdef{\fractStart}{\intcalcAdd{\modulo}{1}}%
  \substring[q]{#1}{\fractStart}{$}%
  \csedef{fractional}{\thestring}%
%% Round and echo
  \RealFPround{\rounded}{\integer.\fractional}{#2}%
  \renewcommand{\ScriptValue}{\RealScriptValue}%
  \ScriptValue{\SI{\rounded}{\prefix#3}}%
}


%%
%% Some computations fit better here in LaTeX than the scripts.
%%

%% Minimum overall PID of each histone type
\FPmin{\result}{\HTwoAPID}{\HTwoBPID}
\FPmin{\result}{\result}{\HThreePID}
\FPmin{\HallMinPID}{\result}{\HFourPID}

\FPeval{\PercentageGenesInHISTOne}{\CoreCodingGenesInHISTOne{}*100/\TotalCoreCodingGenes{}}
