\documentclass[10pt,a4paper,draft]{article}
\usepackage[pdftex]{graphicx}                                           % load images with pdflatex
\usepackage{url}                                                        % nicely formatted URL's
\usepackage{todonotes}                                                  % insert to–do and missing figure items in the document
  \newcommand{\addref}{\todo[color=red!40,size=\tiny]{Add reference}}   % new command for box about missing references
  \newcommand{\rewrite}[2][]{\todo[inline,color=green!40,#1]{#2}}       % new command for boxes of text that needs to be rewritten

\begin{document}
  \section{Introduction}
    %% Complementary introduction on the chromatin organization and how it's dynamic
    %% Explain that despite the fact that there's 4 core histones, there's actually several isoforms of each
    %% Say that no one bothers to explain which one they are using
    %% Comment that maybe the lack of information is because they are so similiar which makes difficult most of the analysis
    %% We are going to annotate all canonical histone genes on (at least GenBank and Ensembl)

    \subsection{Canonical isoforms vs variants}
      %% Characteristics of a variant are the fact they seem to have special functions. Emphasis that some canonical may have special functions whicyh no one bothered to check
      %% variants with extra special nomenclature are outside the histone clusters and have poly-A tails (sometimes)
      %% Complain about the lack of information on the different isoforms
      %% Complain that there's small reason to call some the .1 variant, and the others .2

    \subsection{Special characteristics of the histone genes}
      %% no poly-A tail, a stem-loop instead
      %% great increase on expression during S-phase only
      %% explain expression control of histones
      %% IMAGE - diagram of expression model for cell biologists (blobs representing proteins)

    \subsection{Genome organization}
      %% Talk about the histone clusters and the distribution of each histone
    
      %% IMAGE - diagram of each histone position on the histone cluster

  \section{Canonical core histones}
    %% Consensus for each of them, table with their differences (there's at least 1 error on the Marzulff tables)
    %% table with all gene ids, acession number, official symbols etc
    \subsection{H2A}
    \subsection{H2B}
    \subsection{H3}
      %% I wonder what we'll be saying here
    \subsection{H4}
      %% these should all be the same

  \section{Stem-loop}
    %% Consensus of each stem-loop

  \section{mRNA expression levels}
    %% levels of mRNA of each gistone gene in 2-3 different human cell lines (at least one primary cell line)
\end{document}
