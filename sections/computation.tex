\section{Reproducible research}
\label{sec:reproducible}

	Curation of genomes is an ongoing process, 
	with sequence gaps being completed or corrected,  
	and with annotations being added or updated continuously \addref{some generic curation overview paper}.

	A considerable number of builds and updates of the human genome sequence have been made since 2002 
	so it is unsurprising that 
	\todo{number of sequences added} genes added, 
	\todo{number of sequences removed} removed, 
	and \todo{number of sequences changed} sequences changed  \tref{tab:difference-from-Marzluff02}
	since the last major survey of human canonical histone genes \citep{Marzluff02}.

	Resequencing efforts such as the 1000 genomes initiative have likely  most coding regions have stabilised and 

	In addition to maintaining a current reference, 
	it is also necessary for communities of researchers to feed back information 
	to ensure that reference databases are rich and accurate. 
	In the course of our analyses we have submitted a number of proposals 
	for incorrect, inconsistent or missing annotations of coanonical histone genes. 
	For example, \todo{need a brief example}.
	
	One  process of he continuous curation of genomic databases and it was this
  that led us to publish an updated catalogue of the human histones. The data being
  used in this publication is still being updated. Indeed, it will never stop being
  updated. Since the goal of this publication is an up to date analysis of all histone
  genes, it will become outdated or incorrect once published.
  In addition,
  we are also aware how even the same data can give different results if analysed
  in different ways. For example, \fref{fig:H2A-weblogo} will be different based
  on the alignment algorithm being used such as TCoffee\citep{tcoffee2000},
  ClustalW\citep{clustalw2}, or MultAlin\citep{multalin1988}.

  Since the nature of this publication is a moving target, we have created a set of tools
  to continuously update it based on the latest data that can be found online.
  All sequences were obtained and analysed automatically via a collection of perl
  programs, and we made extensive use of other perl modules, mainly BioPerl\citep{bioperl}.
  When possible, we provided patches to our dependencies, avoiding the infamous
  ``not invented here'' syndrome. This not only makes it easier for colleagues to
  reuse it, whose work may have nothing to do with histones, but also to
  increase its availability.

  At the core of our work is a perl program (bp\_genbank\_ref\_extractor) which
  we incorporated into the Bio-EUtilities distribution since version 1.73. This program
  downloads all sequences --- genomic, transcript, and protein --- related to a search
  result in Entrez gene. Several options have been implemented to select specific genome
  builds, or regions of the genome through offsets from the search results. For example,
  it is possible to download the 500bp upstream of all genes in a search result.

  %% genes may not change a lot but this a good framework to still study the promotor

  For the specific case of human histones, we may have reached a point where the amount
  of changes over time is too small to justify ths work. However, this project was
  designed with reuse in mind, and a catalogue for other organisms can be done with a
  one line change. We have already done this for \textit{Mus musculus} and
  \textit{Gallus gallus} which are of importance in our institute.
  We reckon that our code could also be reused for other families of genes, such as our
  previous catalogue of Snf2 family members\citep{andrew-snf2-catalogue}.

%    We reason that the use of mature software development tools, could be used in other quantitative biology.
  Following the principles of reproducible research
  \citep{reproducible-research-bioinformatics, reproducible-research-law}, we made
  not only all of the source code freely accessible online, but also the instructions
  for automatic builds with SCons\citep{SCons2005}
  (see \url{https://github.com/af-lab/histone-catalog}). The use of such build systems is
  standard practice in software development and is becoming more common in the field
  of computer sciences. But we see this publication with the results in human readable
  form as the goal, not its raw results. Therefore, the final target of our build system is
  the PDF for this publication.

  %% How can we talk about the idea at
  %% http://juretriglav.si/how-scientific-figures-should-work-in-2014/
  %% if we can't cite URL? A very cool idea was mentioned there and deserves
  %% attribution:
  %%
  %%    A great example is a continuous stream of data from an OpenEHR API: why
  %%    wouldn’t that retrospective study on pancreatic cancer survival using
  %%    drug X be a living research object, updating itself as long as the data
  %%    keeps coming?


  Our interest in making this a live and reproducible publication was the basis of
  our choice for the search and download of sequences. Sequences are download from
  RefSeq through Entrez, instead of HAVANA through Ensembl, only because it is much
  easier to make work in other systems. Entrez has made available E-utilities with
  a SOAP interface which is already part of the Bioperl project.
  On the other hand, the Ensembl search is done by a Lucene server which is not publicly
  accessible. The set up of a Lucene instance in a local system is not a straightforward
  solution.
  In addition, the Ensembl perl API is also not part of the Bioperl project, and not
  even available through CPAN. Its dependency on a severely outdated version of BioPerl
  is also a further complication of its installation, to the point that distribution of
  a virtual machine is recommended.
  All this would make reproducibility
  of our data a much more difficult process by other researchers.


