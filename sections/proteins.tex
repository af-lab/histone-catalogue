\section{The proteins}
  While some of the histone genes encode the same protein, this is not always the
  case. On the case of H3 and H4 the number of unique proteins are low, \HThreeUniqueProteins{}
  and \HFourUniqueProteins{} respectively. For H2A and H2B this number is much higher, \HTwoAUniqueProteins{}
  and \HTwoBUniqueProteins{}.

  The numbering of these proteins amino acids also has a catch. The first methionine
  is cleaved after expression and as such, it is never counter when numbering them.

  Most of the PTM are in the histone tails but a few are within the histone fold (H3 T45 for example)

  \subsection{Histone Fold Domain}
    \todo[inline]{Is histone protein structure too much off topic?}

    All histone proteins have a common motif named the Histone Fold Domain (HFD).
    It is composed of a long $\alpha$ helix and two loops on its side that fold
    back two other shorter $\alpha$ helices. This structure allows the histones
    to dimerize with their pair (H3--H4 and H2A--H2B) and form the histone
    octamer. The central $\alpha$ helix of each dimer pair fits in the gap between
    the $\alpha$1 and $\alpha$2 helices crossing the pair $\alpha$2 helix. In
    addition to these, H3 and H2B display a short $\alpha$N and $\alpha$C helix
    on their terminals.

    The HFD is common to all histone and shared with other
    proteins such as the CENP--TWSC and NF--Y complexes. All of these are nuclear
    proteins that interact with DNA. In the case of NF--Y, these even replace
    the H2A--H2b histones during active conditions.

    Whether these proteins actually ever happen to duplicate the nucleosome functions...


  \subsection{H2A}
    \begin{TableAndFigure*}
      \captionof{table}{histone H2A protein consensus}
      \label{tab:H2A-consensus}
      \input{results/table-H2A-align}

      \includegraphics[width=\textwidth]{seqlogo_H2A.pdf}
      \captionof{figure}{WebLogo of all H2A sequences. The bottom line are the amino acids
                         of H2AFJ whose sequence is different from any of the H2A proteins}
      \label{fig:H2A-weblogo}
    \end{TableAndFigure*}

    \todo[inline]{comment on H2AFJ}

  \subsection{H2B}
    \begin{TableAndFigure*}
      \captionof{table}{histone H2B protein consensus}
      \label{tab:H2B-consensus}
      \input{results/table-H2B-align}

      \includegraphics[width=\textwidth]{seqlogo_H2B.pdf}
      \captionof{figure}{WebLogo of all H2B sequences.}
      \label{fig:H2B-weblogo}
    \end{TableAndFigure*}

  \subsection{H3}
    \begin{TableAndFigure*}
      \captionof{table}{histone H3 protein consensus}
      \label{tab:H3-consensus}
      \input{results/table-H3-align}

      \includegraphics[width=\textwidth]{seqlogo_H3.pdf}
      \captionof{figure}{WebLogo of all H3 sequences.}
      \label{fig:H3-weblogo}
    \end{TableAndFigure*}

  \subsection{H4}
    %% these should all be the same but they are not. And there's an H4 in HIST4
    %% which is also present in mouse, which means has been conserved  in evolution
    With the exception of HIST1H4G, all H4 genes seem to encode the same protein. Even HIST4H4 which is the
    only gene on that ``cluster''.
    \begin{TableAndFigure*}
      \captionof{table}{histone H4 protein consensus}
      \label{tab:H4-consensus}
      \input{results/table-H4-align}

      \includegraphics[width=\textwidth]{seqlogo_H4.pdf}
      \captionof{figure}{WebLogo of all H4 sequences.}
      \label{fig:H4-weblogo}
    \end{TableAndFigure*}

