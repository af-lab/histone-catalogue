\section{Histone proteins}
  Most depictions of chromatin imply that
  canonical histone protein types behave as a single protein,
  often referred to as ``wild type''.
  The assumption is based on the relatively high identity of histone sequences
  between isoforms and species,
  and the focus on functional distinctions of histone variants.
  However, the strong selection pressure on histone amino acid
  sequences \citep{birth-death-review},
  the roles of specific amino acid differences \citep{MazeAllis2014},
  and the structural effects of small variations \citep{KurumizakaCOSB2013}
  all suggest that the minor differences in canonical histone protein isoforms
  could have functional implications.

  Isoform variation in canonical histone types H2A and H2B is most pronounced,
  with \HTwoAUniqueProteins{}~H2A and \HTwoBUniqueProteins{}~H2B distinct polypeptides.
  In contrast, only \HThreeUniqueProteins{}~H3 and \HFourUniqueProteins{}~H4 proteins
  are encoded. Although most variation is by amino acid substitution, 
  some H2A and H2B isoforms show C-terminal length differences.

  The nomenclature for histone protein isoforms follows directly
  from the gene names \citep{Marzluff02},
  and supersedes the earlier nomenclature with forward slash which used different isoform letters
  \citep{AlbigGenomics1997,AlbigHumangen1997}.
  Human canonical histone polypeptides have typically been numbered
  with omission of the N-terminal methionine
  since this is likely to be removed because most sequences have
  a small hydrophilic amino acid as the second residue \citep{XiaoPeiBiochem2010}.
  This convention predates the Human Genome Variation Society (HGVS) recommendation
  to include the initial methionine as residue 1.
  We have omitted the N-terminal methionine on figures, alignments 
  and amino-acid numbering for consistency.
  
  %% if we want to change our stance in this, we only need to uncomment
  %% 3 lines in MyLib.pm, in the routine that reads the sequence files.

  \subsection{Canonical H2A isoforms}
    Canonical H2A genes encode \HTwoAUniqueProteins{}~different H2A
    protein isoforms with at least \FPround{result}{\HTwoAPID}{0}\result\% identity (\tref{tab:H2A-consensus})
    These are separable by TAU PAGE into two bands identified as H2A.1 and H2A.2.
    TAU PAGE distinguishes leucine or methionine at
    residue 52 \citep{FranklinZweidler1977,Zweidler1977},
    implying there are up to 8 different protein sequences in H2A.1 and 2 in H2A.2.
    There is no concordance between H2A.1 and H2A.2
    and the location of their isoform-encoding genes in the histone gene clusters.
    HIST2 contains genes encoding isoforms with both Leu52 and Met52,
    while HIST1, HIST2, and HIST3 all contain genes encoding H2A isoforms with Leu52.
    No functional distinction between H2A.1 and H2A.2 has yet been reported.

    Excluding \textit{HIST1H2AA} discussed above,
    the major sites of difference in canonical H2A are
    serine or threonine at residue 17,
    alanine or serine at residue 41,
    lysine or arginine at residue 100,
    and the C-terminal residues from residue 125 onwards (\tref{tab:H2A-consensus}).
    All these sites have implications for post-translational modifications.

    \begin{table}
      \caption{%
        Canonical H2A protein isoforms.  Upper panel shows isoforms
        relative to the most common protein sequence.  Lower panel
        shows sequence logo of all isoforms aligned with invariant
        residues in grey.
      }
      \label{tab:H2A-consensus}
      \input{results/table-H2A-proteins-align}
      \includegraphics[width=\textwidth]{seqlogo_H2A_proteins.pdf}
    \end{table}

  \subsection{Canonical H2B isoforms}
    Canonical H2B has the most canonical isoforms (\tref{tab:H2B-consensus})
    despite these \HTwoBUniqueProteins{}~unique proteins 
    with at least \FPround{result}{\HTwoBPID}{0}\result\% identity
    all migrating together in TAU PAGE.

    There is significant variability between isoforms in the N-terminal region
    and this is one of the most variable sites between canonical
    core histones of different species.
    Human H2B has a unique and distinctive N-terminal proline-acidic-proline motif (PEP/PDP)
    followed by a very variable residue with either alanine, serine,
    threonine or valine (\tref{tab:H2B-consensus}).

    Excluding \textit{HIST1H2BA} discussed above,
    the remaining isoform differences are mainly the chemically
    similar valine or isoleucine at residue 40,
    and serine to glycine and alanine at residues 76 and 125 respectively
    which both have post-translational modification implications.
    Variation in H2B isoform levels between multiple cancer cell
    lines has been observed \citep{Molden2015},
    although the functional implications are unknown.

    \begin{table}
      \caption{%
        Canonical H2B protein isoforms.  Upper panel shows isoforms
        relative to the most common protein sequence.  Lower panel
        shows sequence logo of all isoforms aligned with invariant
        residues in grey.
      }
      \label{tab:H2B-consensus}
      \input{results/table-H2B-proteins-align}
      \includegraphics[width=\textwidth]{seqlogo_H2B_proteins.pdf}
    \end{table}

  \subsection{Canonical H3 isoforms}
    Canonical H3 genes encode only \HThreeUniqueProteins{}~different
    canonical isoforms (\tref{tab:H3-consensus}).
    The majority of H3 genes are in the HIST1 cluster and encode a
    single polypeptide sequence \citep{Ederveen2011}
    whereas the three canonical H3 genes in HIST2 encode a different isoform
    with the interesting difference of serine instead of cysteine at residue 97
    that is separable by TAU PAGE \citep{FranklinZweidler1977}.
    By apparent coincidence this means HIST1-encoded canonical H3 isoforms are
    identified as H3.1 and the HIST2-encoded copies are H3.2.
    As discussed above, \textit{HIST3H3} appears to be 
    a largely testes specific variant H3T
    with four amino acid differences that has been assigned as H3.4 \citep{Talbert2012}
    even though it likely migrates with H3.1 in TAU PAGE.

    \begin{table}
      \caption{%
        Canonical H3 protein isoforms.  Upper panel shows isoforms
        relative to the most common protein sequence.  Lower panel
        shows sequence logo of all isoforms aligned with invariant
        residues in grey.
      }
      \label{tab:H3-consensus}
      \input{results/table-H3-proteins-align}
      \includegraphics[width=\textwidth]{seqlogo_H3_proteins.pdf}
    \end{table}

  \subsection{Canonical H4 isoforms}
    H4 is the most homogeneous of all canonical core histones,
    with all but one of the \HFourCodingGenes{}~genes encoding
    identical proteins (\tref{tab:H4-consensus}).
    These genes are located across HIST1, HIST2 and the isolated
    \textit{HIST4H4} comprising HIST4.
    The divergent \textit{HIST1H4G} gene in the middle of the HIST1 cluster
    encodes an isoform with 15 amino acid differences and a deletion of the C-terminal 5 residues.
    This gene is annotated as being transcribed so merits further investigation.

    \begin{table}
      \caption{%
        Canonical H4 protein isoforms.  Upper panel shows isoforms
        relative to the most common protein sequence.  Lower panel
        shows sequence logo of all isoforms aligned with invariant
        residues in grey.
      }
      \label{tab:H4-consensus}
      \input{results/table-H4-proteins-align}
      \includegraphics[width=\textwidth]{seqlogo_H4_proteins.pdf}
    \end{table}
