  \begin{supplement}
    \caption{
        Changes in human canonical core histone gene catalogue
        annotated in NCBI RefSeq obtained on \protect\printdate{\SequencesDate{}}
        compared to \citet{Marzluff02}.
    }
    \label{tab:difference-from-Marzluff02}
    \input{results/table-reference_comparison}
  \end{supplement}

  \newpage
  %% This is a very long table (xtabular), that spans multiple pages.  TeX
  %% standard floats are unable to handle it.
  \captionof{supplement}{%
    Catalogue of annotated human canonical core histone genes, transcripts,
    and encoded proteins.
    Data obtained from NCBI RefSeq \citep{OLearyRefseq2016} on \protect\printdate{\SequencesDate{}}.
  }
  \label{tab:histone-catalogue}
  \input{results/table-histone_catalogue}

  \newpage
  \begin{supplement}
    \centering
    \caption{
        Sequence logo for the human canonical H2A gene coding regions
        listed in table \tref{tab:histone-catalogue}.
        Initiator codon ATG and stop codon are omitted.
    }
    \label{fig:h2a-histone-gene-variation}
    \includegraphics[width=1.0\textwidth]{figs/seqlogo_H2A_cds.pdf}
  \end{supplement}
  \newpage
  \begin{supplement}
    \centering
    \caption{
        Sequence logo for the human canonical H2B gene coding regions
        listed in table \tref{tab:histone-catalogue}.
        Initiator codon ATG and stop codon are omitted.
    }
    \label{fig:h2b-histone-gene-variation}
    \includegraphics[width=1.0\textwidth]{figs/seqlogo_H2B_cds.pdf}
  \end{supplement}
  \newpage
  \begin{supplement}
    \centering
    \caption{
        Sequence logo for the human canonical H3 gene coding regions
        listed in table \tref{tab:histone-catalogue}.
        Initiator codon ATG and stop codon are omitted.
    }
    \label{fig:h3-histone-gene-variation}
    \includegraphics[width=1.0\textwidth]{figs/seqlogo_H3_cds.pdf}
  \end{supplement}
  \newpage
  \begin{supplement}
    \centering
    \caption{
        Sequence logo for the human canonical H4 gene coding regions
        listed in table \tref{tab:histone-catalogue}.
        Initiator codon ATG and stop codon are omitted.
    }
    \label{fig:h4-histone-gene-variation}
    \includegraphics[width=1.0\textwidth]{figs/seqlogo_H4_cds.pdf}
  \end{supplement}

  \newpage
  \begin{supplement}
    \caption{
        Annotated histone variants.
        HGNC histone family names \citep{HGNC2015}, commonly used protein names,
        and nomenclature proposal of \citet{Talbert2012}.
    }
    \centering
    \begin{tabularx}{\linewidth}{l l l >{\raggedright\arraybackslash}X}
      \toprule
      Family & Common name & \citet{Talbert2012} & Notes \\
      \midrule
      H2AFB & H2A.Bbd & H2A.B & equivalent to H2A.L \\
      H2AFJ & H2A.J & H2A.J & at HIST4 locus \\
      H2AFV & H2A.Z-2 & H2A.Z.2 & - \\
      H2AFX & H2AX/H2A.X & H2A.X & - \\
      H2AFY & macroH2A/mH2A & macroH2A & - \\
      H2AFZ & H2A.Z & H2A.Z.1 & - \\
      H2BFM & ? & H2B.M & homologous to H2B.W, X-linked\\
      H2BFS & - & - & pseudogene \\
      H2BFWT & ? & H2B.W & testes specific, X-linked \\
      H2BFX & - & - & pseudogene \\
      HIST1H2BA & TH2B/TSH2B & TS H2B.1 & testes specific \\
      HIST3H3 & H3T & H3.4 & testes specific \\
      H3F3 & H3.3 & H3.3 & euchromatin related \\
      CENPA & CENP-A & - & centromere-specific \\
      \bottomrule
    \end{tabularx}
  \end{supplement}

  \newpage
  \begin{supplement}
    \caption{
        Mean synonymous and non-synonymous distances between pairs of canonical core histone genes.
        Mean over all pairwise comparisons of non-synonymous ($d_N$) and synonymous ($d_S$) 
        nucleotide substitutions and mean of $d_N/d_S$ ratios for canonical core histone coding regions
        computed using \citet{GoldmanYang1994} model implemented in
        \texttt{codeml} from the PAML package \citep{PAML2007}.
    }
    \label{tab:histone-gene-differences}
    \centering
    \begin{tabular}{l l l l}
      \toprule
      \null & \centercell{$d_N$} & \centercell{$d_S$} & \centercell{$d_N/d_S$} \\
      \midrule
      H2A & \MeanHTwoAdN  & \MeanHTwoAdS  & \MeanHTwoAdNdS \\
      H2B & \MeanHTwoBdN  & \MeanHTwoBdS  & \MeanHTwoBdNdS \\
      H3  & \MeanHThreedN & \MeanHThreedS & \MeanHThreedNdS \\
      H4  & \MeanHFourdN  & \MeanHFourdS  & \MeanHFourdNdS \\
      \bottomrule
    \end{tabular}
  \end{supplement}

  \newpage
  \captionof{supplement}{%
    Codon usage frequency for each amino acid
    and canonical core histone type.
  }
  \label{tab:histone-gene-codonusage}
  \input{results/table-codon_usage}

  \newpage
  \captionof{supplement}{%
    Catalogue of annotated human core histone variant genes, transcripts and encoded proteins.
    Catalogue of annotated human core histone variant genes, transcripts,
    and encoded proteins.
    Data obtained from NCBI RefSeq \citep{OLearyRefseq2016} on \protect\printdate{\SequencesDate{}}.
  }

  \label{tab:variant-catalogue}
  \input{results/table-variant_catalogue}

  \newpage
  \begin{supplement}
    \caption{
        Variations in canonical core histone gene annotations
        compared to expectation of single exon, single transcript, and stem-loop.
        Data obtained from NCBI RefSeq \citep{OLearyRefseq2016} on \protect\printdate{\SequencesDate{}}.
    }
    \label{tab:curation-anomalies}
    \begin{itemize}
    \input{results/histone_insanities.tex}
    \end{itemize}
  \end{supplement}
