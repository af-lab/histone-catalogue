\section{Supplementary Figures}

\beginsupplement

%% Supplementary tables and figures
\onecolumn

\newpage
\captionof{table}{Differences between the current human histone genes and \citep{Marzluff02}.}
\label{tab:difference-from-Marzluff02}
\input{results/table-reference_comparison}

  %% this is a very long table (ctabular) that does not play nice with two columns
  %% so we place it at the end. See why it was a problem at
  %% http://tex.stackexchange.com/questions/101791/float-a-wide-ctabular-in-two-columns
  %% We could split the table in two parts in which case we could use begin{table*}
\newpage
\captionof{table}{Catalogue of all annotated human canonical core histone gene, transcript and protein IDs}
\label{tab:histone-catalogue}
\input{results/table-histone_catalogue}

\begin{figure*}
	\centering
	\subtop[H2A]{
		\includegraphics[width=1.0\textwidth]
		{seqlogo_H2A_cds.pdf}
		\label{fig:H2A-cds-weblogo}
	  }
	\hfill
	\subtop[H2B]{
		\includegraphics[width=1.0\textwidth]
		{seqlogo_H2B_cds.pdf}
		\label{fig:H2B-cds-weblogo}
	}
	\subtop[H3]{
	\includegraphics[width=1.0\textwidth]
		{seqlogo_H3_cds.pdf}
		\label{fig:H3-cds-weblogo}
	}
	\hfill
	\subtop[H4]{
		\includegraphics[width=1.0\textwidth]
		{seqlogo_H4_cds.pdf}
		\label{fig:H4-cds-weblogo}
	}
	\captionof{figure}{Sequence logos for canonical coding histone genes from \tref{tab:histone-catalogue}. 
	showing coding regions of A. H2A, B. H2B, C. H3 and D. H4.}
	\label{fig:histone-gene-variation}
	\end{figure*}

\newpage
\begin{table*}
	\caption{HGNC recognised histone variant family stem names, commonly used protein names 
	and names for improved consistency based on proposal by Talbert \textit{et al} \citep{Talbert2012}. 
	\textit{HIST1H2BA} and \textit{HIST3H3} are included as discussed above.}
	\label{tab:histone-variant-families}
	\centering
	\begin{tabular}{l l l l}
	\toprule
	Family & Protein & Talbert \textit{et al} & Notes \\
	\midrule
	H2AFB & H2A.Bbd & H2A.B & equivalent to H2A.L \\
	H2AFJ & H2A.J & H2A.J & at HIST4 locus \\
	H2AFV & H2A.Z-2 & H2A.Z.2 & - \\
	H2AFX & H2AX/H2A.X & H2A.X & - \\
	H2AFY & macroH2A/mH2A & macroH2A & - \\
	H2AFZ & H2A.Z & H2A.Z.1 & - \\
	H2BFM & ? & H2B.M & homologous to H2B.W, X-linked\\
	H2BFS & - & - & pseudogene \\
	H2BFWT & ? & H2B.W & testes specific, X-linked \\
	H2BFX & - & - & pseudogene \\
	HIST1H2BA & TH2B/TSH2B & H2B.1 & testes specific \\
	HIST3H3 & H3T & H3.4 & testes specific \\
	H3F3 & H3.3 & H3.3 & euchromatin related \\
	CENPA & CENP-A & - & centromere-specific \\
	\bottomrule
	\end{tabular}
\end{table*}

\newpage
\begin{table*}
	\caption{
	Sequence differences between canonical histone genes.
	Average counts of synonymous (Sd) and non-synonymous (Sn) nucleotide substitutions 
	between all pairwise comparisons of canonical histone coding regions by type 
	and average ds/dn ratios for the pairwise comparisons, 
	calculated by SNAP.}
	\label{tab:histone-gene-differences}
	\centering
	\begin{tabular}{r r r r}
		\toprule
		\null & Sd & Sn & ds/dn \\
		\midrule
		H2A	& 52.7 & 5.8 & 75.1 \\
		H2B	& 36.9 & 5.5 & 58.7 \\
		H3	& 57.1 & 1.8 & 214.4 \\
		H4	& 49.2 & 3.9 & 204.7 \\
		\bottomrule
	\end{tabular}
\end{table*}

\newpage
\begin{table*}
	\caption{
	Codon usage percentages for all canonical histone coding regions 
	calculated by CUSP.}
	\label{tab:histone-gene-codonusage}
	\centering
	\begin{tabular}{r r r r}
		\toprule
		\null &  & ds/dn \\
		\midrule
		Ala	&  \\
		Arg	& 36.9 & 5.5 & 58.7 \\
		H3	& 57.1 & 1.8 & 214.4 \\
		H4	& 49.2 & 3.9 & 204.7 \\
		\bottomrule
	\end{tabular}
\end{table*}

\newpage
\captionof{table}{Catalogue of human variant core histone gene, transcript and protein IDs}
\label{tab:variant-catalogue}
\input{results/table-variant_catalogue}

\newpage
\input{results/histone_insanities.tex}
