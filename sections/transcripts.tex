\section{Histone transcripts}
  %% We need to restrict ourselves to describing properties related to the catalogued sequences
  %% and to avoid getting sucked into being encyclopedic, even if it might be useful
  %% More detailed other stuff could go in your thesis some other way. Not sure how?
  %% I think all we can say here is that the promoters and stem loops are much the same in all histones?
  
  %% General description of histone transcripts
  %% Enhancers (refer only), promoter, mRNA capping? (refer), no introns, no polyA tails (refer), cell cycle regulation (refer)
	Canonical core histone genes are transcribed by RNA polymerase II 
	and have a methyl guanosine cap \addref{Marzluff NatRevGen 2008}
	but, unlike almost all other protein coding genes including histone variants, 
	they do not have introns (\tref{tab:typical-histone-differences}) 
	and only 1-5\% of canonical histone transcripts are 3' polyadenylated \addref{YangChen GenomeBiol 2011}

	Instead, canonical histone transcripts contain a 3' stem-loop in their mRNA secondary structure 
	that is recognised by specialised RNA processing and stabilisation complexes.
	The stem-loop is encoded \StemLoopStart{} to \StemLoopEnd{} base pairs after the stop codon 
	and has a highly conserved sequence and structure (\fref{fig:stem-loop-seqlogo}) \citep{stem-loop-structure}.
	The stem-loop interacts with the Stem-Loop Binding Protein (SLBP) 
	which is up-regulated 10-20 fold during S~phase to stabilise the histone mRNA \citep{SLBP-regulation}, 
	in order to lengthen the canonical histone mRNA half lives 
	to assist the increased histone protein requirement for packaging the newly duplicated genome.
	This is followed by a purine-rich Histone Downstream Element (HDE) 
	that interacts with U7 snRNA to direct efficient 3' end processing \citep{HDE-sequence}.

	The histone variant H2AX is unusual in having 3' transcript properties reflecting
	both canonical and variant histone features \citep{HTwoAX-transcripts}, including a canonical histone stem-loop. 
	There are also some canonical histone gene copies that are annotated with two transcripts 
	differing in whether they have stem-loop or polyadenylation signals (\tref{tab:histone-catalogue}). 

	Overall there is estimated to be a 35 fold increase in mammalian canonical histone transcripts in S phase,
	principally as a result of a 10 fold increase in mRNA stabilisation via stem-loop dependent mechanisms 
	and a 3-5 fold up-regulation in canonical histone gene transcription \addref{Marzluff NatRevGen 2008}. 
	This transcription is regulated by cell-cycle dependent phosphorylation of the histone-specific NPAT coactivator 
	and interaction with the accessory protein FLASH, 
	resulting in assembly of histone locus bodies localising factors responsible for canonical histone transcription and processing.

    \begin{figure*}
      \centering
      \includegraphics[width=\textwidth]{seqlogo_stem_loops.pdf}
      \captionof{figure}{seqlogo of human stem-loop}
      \label{fig:stem-loop-seqlogo}
    \end{figure*}
