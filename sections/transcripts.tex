\section{Histone transcripts}
  
	Canonical core histone genes are transcribed by RNA polymerase II 
	and have a methyl guanosine cap \citep{MarzluffNatRevGen2008}
	but, unlike almost all other protein coding genes including histone variants, 
	they do not have introns and are not polyadenylated (\tref{tab:typical-histone-differences}).

	Canonical histone gene transcription is regulated 
	by cell-cycle dependent phosphorylation of the histone-specific NPAT coactivator 
	and interaction with the accessory protein FLASH, 
	resulting in assembly of histone locus bodies 
	coordinating factors responsible for canonical histone transcription and processing \citep{MarzluffNatRevGen2008,RattrayMueller2012,Hoefig2014}.
	Overall there is estimated to be a 35 fold increase in mammalian canonical histone transcripts in S phase,
	principally as a result of a 10 fold increase in mRNA stabilisation via stem-loop dependent mechanisms 
	and a 3-5 fold up-regulation in canonical histone gene transcription \citep{HarrisMCB1991}.

	This post-translation regulation of canonical histone transcripts 
	is achieved by a stem-loop encoded in their mRNA 3' untranslated region 
	that is recognised by specialised RNA processing and stabilisation complexes.
	The stem-loop is encoded from \StemLoopStart{} to \StemLoopEnd{} base pairs after the stop codon 
	and has a highly conserved sequence and structure (\fref{fig:stem-loop-seqlogo} \citep{stem-loop-structure}).
	The RNA stem-loop structure is bound by the Stem-Loop Binding Protein (SLBP) 
	which is up-regulated 10-20 fold during S~phase to stabilise histone mRNAs \citep{SLBP-regulation}. 
	Immediately downstream of the stem-loop a purine-rich Histone Downstream Element (HDE) 
	interacts with U7 snRNA to direct efficient 3' end processing (\fref{fig:HDE-seqlogo}) \citep{HDE-sequence}.
  \begin{figure*}
    \centering
    \subtop[stem loop]{
      \includegraphics{seqlogo_stem_loops.pdf}
      \label{fig:stem-loop-seqlogo}
    }
    \hfill
    \subtop[HDEs]{
      \includegraphics{seqlogo_HDEs.pdf}
      \label{fig:HDE-seqlogo}
    }
    \captionof{figure}{Sequence logos for stem-loop and HDEs.}
  \end{figure*}

	Stabilisation and processing lengthen canonical histone mRNA half lives 
	and contribute to increased histone translation efficiency 
	enabling rapid production of canonical histones to package the newly duplicated genome.

	Although the vast bulk of canonical histone transcripts appear to depend on this mechanism, 
	1-5\% of canonical histone transcripts are 3' polyadenylated \addref{YangGenomeBiol2011} 
	and some genes have annotations indicating two transcripts 
	differing in whether they have stem-loop or polyadenylation signals (\tref{tab:histone-catalogue}). 
	Almost all histone variant transcripts lack the stem-loop and are polyadenylated 
	with the exception of transcripts with 3' properties alternatively displaying
	variant or canonical features including a canonical histone stem-loop \citep{HTwoAX-transcripts}. 


