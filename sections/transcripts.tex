\section{Histone transcripts}
  %% We need to restrict ourselves to describing properties related to the catalogued sequences
  %% and to avoid getting sucked into being encyclopedic, even if it might be useful
  %% More detailed other stuff could go in your thesis some other way. Not sure how?
  %% I think all we can say here is that the promoters and stem loops are much the same in all histones?
  
  %% General description of histone transcripts
  %% Enhancers (refer only), promoter, mRNA capping? (refer), no introns, no polyA tails (refer), cell cycle regulation (refer)
	Canonical core histone genes are transcribed by RNA polymerase II 
	and have a methyl guanosine cap \addref{Marzluff NatRevGen 2008}
	but, unlike almost all other protein coding genes including histone variants, 
	they do not have introns and are not polyadenylated (\tref{tab:typical-histone-differences}).

	Canonical histone gene transcription is regulated 
	by cell-cycle dependent phosphorylation of the histone-specific NPAT coactivator 
	and interaction with the accessory protein FLASH, 
	resulting in assembly of histone locus bodies 
	coordinating factors responsible for canonical histone transcription and processing \addref{Marzluff NatRevGen2008}.
	Overall there is estimated to be a 35 fold increase in mammalian canonical histone transcripts in S phase,
	principally as a result of a 10 fold increase in mRNA stabilisation via stem-loop dependent mechanisms 
	and a 3-5 fold up-regulation in canonical histone gene transcription. 

	This post-translation regulation of canonical histone transcripts 
	is achieved by a stem-loop encoded in their mRNA 3' untranslated region 
	that is recognised by specialised RNA processing and stabilisation complexes.
	The stem-loop is encoded from \StemLoopStart{} to \StemLoopEnd{} base pairs after the stop codon 
	and has a highly conserved sequence and structure (\fref{fig:stem-loop-seqlogo}) \citep{stem-loop-structure}).
	This RNA structure is bound by the Stem-Loop Binding Protein (SLBP) 
	which is up-regulated 10-20 fold during S~phase to stabilise histone mRNAs \citep{SLBP-regulation}. 
	Stabilisation lengthens canonical histone mRNA half lives 
	to provide for increased histone translation capacity to package the newly duplicated genome.
	Immediatley downstream of the stem-loop a purine-rich Histone Downstream Element (HDE) 
	that interacts with U7 snRNA to direct efficient 3' end processing \citep{HDE-sequence}.

	Although the vast bulk of canonical histone transcripts appear to depend on this mechanism, 
	1-5\% of canonical histone transcripts are 3' polyadenylated \addref{YangChen GenomeBiol 2011} 
	and some genes have annotations indicating two transcripts 
	differing in whether they have stem-loop or polyadenylation signals (\tref{tab:histone-catalogue}). 
	Almost all histone variant transcripts lack the stem-loop and are polyadenylated. 
	However, the H2AX variant is unique in having transcripts with 3' properties alternatively displaying
	variant or canonical features including a canonical histone stem-loop \citep{HTwoAX-transcripts}. 

  \begin{figure*}
    \centering
    \subtop[stem loop]{
      \includegraphics{seqlogo_stem_loops.pdf}
      \label{fig:stem-loop-seqlogo}
    }
    \hfill
    \subtop[HDEs]{
      \includegraphics{seqlogo_HDEs.pdf}
      \label{fig:HDE-seqlogo}
    }
    \captionof{figure}{Sequence logos for stem-loop and HDEs.}
  \end{figure*}

