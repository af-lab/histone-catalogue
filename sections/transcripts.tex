\section{Histone transcripts}

  Canonical core histone transcripts are among the few protein coding messages that are not polyadenylated.
  Instead they rely on a unique 3' UTR stem-loop structure
  for regulation of transscript stability.
  They do have a 5' 7--methyl--guanosine cap \citep{MarzluffNatRevGen2008}.

  Canonical core histone gene transcription is regulated
  by cell-cycle dependent phosphorylation of the histone-specific
  Nuclear Protein of the Ataxia-Telangiectasia locus (NPAT) coactivator
  and interaction with the accessory protein
  FADD-Like interleukin-1\textbeta{}-converting
  enzyme/caspase-8-ASsociated Huge protein (FLASH),
  resulting in assembly of histone locus bodies
  coordinating factors responsible for transcription and processing
  \citep{MarzluffNatRevGen2008,RattrayMueller2012,Hoefig2014}.
  Variability is observed in canonical core histone isoform gene transcription,
  both by analysis of non-polyadenylated transcripts \citep{YangGenomeBiol2011}
  and RNA polymerase II promoter occupancy \citep{Ederveen2011}.

  Overall there is estimated to be a 35 fold increase in mammalian
  canonical histone transcripts during S phase,
  principally as a result of a 10 fold increase in mRNA stabilisation
  via stem-loop dependent mechanisms,
  and a 3--5 fold up-regulation in canonical core histone gene transcription \citep{HarrisMCB1991}.

  This post-transcriptional regulation is achieved by
  a stem-loop encoded in the mRNA 3' untranslated region
  that is recognised by spliceosome-related RNA
  processing and stabilisation complexes \citep{stem-loop-structure}.
  The start location of the annotated stem-loops in human canonical histone transcripts
  ranges from \StemLoopStartMin{} to \StemLoopStartMax{} bp after the stop codon
  with a modal value of \StemLoopStartMode{} bp.
  The sequence logo of aligned stem-loops confirms that the stem-loop is
  highly conserved (\fref{fig:stem-loop-seqlogo}).

  The RNA stem-loop structure is bound by the Stem-Loop Binding Protein (SLBP)
  which is up-regulated 10--20 fold during S~phase to stabilise
  histone mRNAs \citep{SLBP-regulation}.
  Immediately downstream of the stem-loop a purine-rich Histone Downstream Element (HDE)
  interacts with U7 snRNA to direct efficient 3' end processing.
  Although this is not an annotated feature of histone genes,
  alignment of the canonical HDE \citep{HDE-sequence} to all canonical histone genes
  shows the modal location of the HDE is
  \HDEsDistanceFromStemLoopMode{} bp downstream of the stem-loop.
  The sequence logo confirms that this feature is also highly conserved (\fref{fig:HDE-seqlogo}).

  \begin{figure}
    \centering
    \subtop[Stem-loop]{
      \includegraphics{figs/seqlogo_stem_loops.pdf}
      \label{fig:stem-loop-seqlogo}
    }
    \hfill
    \subtop[HDE]{
      \includegraphics{figs/seqlogo_HDEs.pdf}
      \label{fig:HDE-seqlogo}
    }
    \caption{%
      Sequence logos for
      \subcaptionref{fig:stem-loop-seqlogo} annotated stem-loops and
      \subcaptionref{fig:HDE-seqlogo}  Histone Downstream Elements (HDEs)
      identified by homology
      for all canonical core histone gene 3' untranslated regions (UTRs).
    }
  \end{figure}

  Stabilisation and processing extend the half life of canonical core histone \mbox{mRNAs}
  during S phase and contribute to increased histone translation efficiency,
  enabling rapid production of histones to package the newly duplicated genomes.

  Although the vast bulk of canonical core histone transcripts
  appear to be regulated by this mechanism,
  1-5\% of transcripts are found to be 3' polyadenylated \citep{YangGenomeBiol2011}
  and some genes have annotations indicating two transcripts
  differing in whether they have stem-loop or polyadenylation signals (\tref{tab:curation-anomalies}).

  Core histone variant transcripts lack a 3' stem-loop and are
  polyadenylated in the same way as most protein coding genes.
  The exception is H2AX, which has alternatively processed transcripts
  exhibiting both the stem-loop characteristic of a canonical core histone
  and a poly(A) tail found on variant core histones \citep{HTwoAX-transcripts,our-H2AX-review}.
