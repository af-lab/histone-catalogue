\section{Introduction}

  Histones are among the most abundant proteins in eukaryotic cells
  and contribute up to half the mass of chromatin \citep{AlbertsMBoC}.
  The core histone types H2A, H2B, H3, and H4
  define the structure and accessibility of the nucleosome
  as the fundamental repeating unit of genome organisation
  around which the DNA is wrapped \citep{Luger1997structure}.
  In addition, the many chemically reactive sidechains of histones
  are post-translationally modified
  as a nexus for signalling and heritable epigenetics \citep{Kouzarides2007}.

  Core histones are delineated as either canonical or variant based on
  their gene location, expression characteristics,
  and functional roles (\tref{tab:typical-histone-differences}).
  Canonical core histones contribute the majority of proteins to
  the bulk structure and generic function of chromatin,
  and are encoded by \TotalCoreGenes{} genes in \NumberOfClusters{}
  clusters named HIST1-HIST\NumberOfClusters{} in the human genome,
  of which \TotalCoreCodingGenes{} are coding genes and \TotalCorePseudoGenes{}
  are pseudogenes (\tref{tab:histone-gene-count}).

  \begin{table}
    \caption{Properties distinguishing canonical and variant core histone proteins.}
    \label{tab:typical-histone-differences}
    \centering
    \begin{tabular}{l l l}
      \toprule
      \null                     & Canonical             & Variants \\
      \midrule
      Expression timing         & Replication dependent & Replication independent \\
      Sequence identity         & High                  & Low \\
      Functional relationships  & Isoforms              & Specialised functions \\
      Transcript stabilisation  & Stem-loop             & poly(A) tail \\
      Gene distribution         & Clusters              & Scattered \\
      \bottomrule
    \end{tabular}
  \end{table}

  \begin{table}
    \caption{Count of human canonical core histone coding genes and pseudogenes
             by histone cluster and type.  \textpsi{} indicates pseudogenes.}
    \label{tab:histone-gene-count}
    \centering
    \begin{tabular}{l *{5}{r!{+}r<{\textpsi}}}
  \toprule
  \null   & \multicolumn{2}{c}{H2A}  & \multicolumn{2}{c}{H2B}
          & \multicolumn{2}{c}{H3}   & \multicolumn{2}{c}{H4}
          & \multicolumn{2}{c}{Total} \\
  \midrule
  HIST1   & \HTwoACodingInHISTOne{}     & \HTwoAPseudoInHISTOne{}
          & \HTwoBCodingInHISTOne{}     & \HTwoBPseudoInHISTOne{}
          & \HThreeCodingInHISTOne{}    & \HThreePseudoInHISTOne{}
          & \HFourCodingInHISTOne{}     & \HFourPseudoInHISTOne{}
          & \CoreCodingGenesInHISTOne{} & \CorePseudoGenesInHISTOne{} \\
  HIST2   & \HTwoACodingInHISTTwo{}     & \HTwoAPseudoInHISTTwo{}
          & \HTwoBCodingInHISTTwo{}     & \HTwoBPseudoInHISTTwo{}
          & \HThreeCodingInHISTTwo{}    & \HThreePseudoInHISTTwo{}
          & \HFourCodingInHISTTwo{}     & \HFourPseudoInHISTTwo{}
          & \CoreCodingGenesInHISTTwo{} & \CorePseudoGenesInHISTTwo{} \\
  HIST3   & \HTwoACodingInHISTThree{}   & \HTwoAPseudoInHISTThree{}
          & \HTwoBCodingInHISTThree{}   & \HTwoBPseudoInHISTThree{}
          & \HThreeCodingInHISTThree{}  & \HThreePseudoInHISTThree{}
          & \HFourCodingInHISTThree{}   & \HFourPseudoInHISTThree{}
          & \CoreCodingGenesInHISTThree{} & \CorePseudoGenesInHISTThree{} \\
  HIST4   & \HTwoACodingInHISTFour{}    & \HTwoAPseudoInHISTFour{}
          & \HTwoBCodingInHISTFour{}    & \HTwoBPseudoInHISTFour{}
          & \HThreeCodingInHISTFour{}   & \HThreePseudoInHISTFour{}
          & \HFourCodingInHISTFour{}    & \HFourPseudoInHISTFour{}
          & \CoreCodingGenesInHISTFour{} & \CorePseudoGenesInHISTFour{} \\
  \addlinespace
  Total   & \HTwoACodingGenes{}       & \HTwoAPseudoGenes{}
          & \HTwoBCodingGenes{}       & \HTwoBPseudoGenes{}
          & \HThreeCodingGenes{}      & \HThreePseudoGenes{}
          & \HFourCodingGenes{}       & \HFourPseudoGenes{}
          & \TotalCoreCodingGenes{}   & \TotalCorePseudoGenes{} \\
  \bottomrule
\end{tabular}

  \end{table}

  Relationships within the histone family have been described using a variety of terminologies
  reflecting biochemical, functional, and genomic perspectives that are briefly described below
  and summarised in \tref{tab:histone-divisions}.

  \afterpage{
    \captionof{table}{Terminology describing histone variation.}
    \label{tab:histone-divisions}
    \definecolor{shadecolor}{gray}{0.9}
    \begin{shaded}
      \begin{description}
        \item[Allelic variants] \hfill \newline
        Copies of canonical histone type genes,
        possibly with different sequences.
        Not located at same exact chromosomal locus as expected for alleles.
        Not histone variants.
        See also ``isoforms''.

        \item[Canonical core histones]\hfill \newline
        Core histones with properties described in \tref{tab:typical-histone-differences}.
        that contribute the majority of core histones in chromatin
        encompassing multiple protein isoforms.
        %% Complement of variant histones.

        \item[Core histones]\hfill \newline
        Histones that form part of the nucleosome core particle wrapping \SI{147}{\bp} of DNA,
        comprising types H2A, H2B, H3, and H4.
        Encompasses both canonical and variant histones.
        %% Complement of linker histones.

        \item[Heteromorphous variants] \hfill \newline
        Core histone variants with distinct function and localisation
        that are readily separated by gel electrophoresis.

        \item[Homomorphous variants] \hfill \newline
        Canonical core histone subtypes
        requiring high resolution separation methods such as TAU PAGE.
        Synonym for ``subtypes''.

        \item[Families] \hfill \newline
        Synonym for ``types''.

        \item[Isoforms] \hfill \newline
        Proteins with high sequence identity and largely equivalent function.
        Functional equivalence has not been demonstrated for canonical histone isoforms.

        \item[Linker histones] \hfill \newline
        Histones binding to linker DNA adjacent to the nucleosome core particle.
        The two linker histone types are H1 and H5.
        %% Complement of core histones.

        \item[Non-allelic variants] \hfill \newline
        Synonym for ``variant histones''. See also ``allelic variants''.

        \item[Replacement histones] \hfill \newline
        Synonym for ``variant histones''.
        Named because they can replace canonical core histones assembled in S phase.

        \item[Replication-dependent histones] \hfill \newline
        Synonym for ``canonical core histones''.
        Named because expression occurs primarily in S phase.

        \item[Replication-independent histones] \hfill \newline
        Synonym for ``variant histones''.
        Named because expression is not predominantly in S phase.

        \item[Subtypes] \hfill \newline
        Canonical core histone type isoforms separable by TAU PAGE
        (e.g. H2A.1 and H2A.2). There is not necessarily functional evidence
        for differences between subtypes.

        \item[Types] \hfill \newline
        Histone proteins sharing sequence homology
        that participate in specific combinations to define the repeating nucleosome structure.
        The 5 histone types are H1, H2A, H2B, H3, and H4.

        \item[Variant histones] \hfill \newline
        Core histones with properties described in \tref{tab:typical-histone-differences}.
        Contribute a minor proportion of histones in chromatin and perform specialised functions.
        %% Complement of canonical core histones.

        \item[Wild type histones] \hfill \newline
        Synonym of ``canonical core histones''.
      \end{description}
    \end{shaded}
  }

  \subsection{Biochemical perspective}

    Abundant histone proteins are readily isolated using their
    highly basic chemical character.
    Successive improvements in fractionation ultimately revealed 5 main histone types
    with nomenclature H1, H2A, H2B, H3, and H4 \citep{nomenclature}.
    An additional H1-related histone H5 is recognised in avian erythrocytes \citep{HFive-review}.

    The demonstration of the nucleosome as the fundamental
    repeating unit of chromatin \citep{Kornberg1974}
    showed that H2A, H2B, H3, and H4 associate as an octamer of two copies each within the
    nucleosome core particle. These four histones are referred to as core histones.
    In contrast, H1 associates with the linker DNA between nucleosome core particles
    and is referred to as a linker histone.
    The somatic H1 isoforms and tissue-specific
    variants are described elsewhere \citep{HarshmanFreitas2013}.

    Arginine and lysine content was used as an early distinction between the histones \citep{ElginWeintraub1975}.
    The H1 linker histone has a low arginine/lysine ratio of
    \FPround{\result}{\LinkerArgLysRatio}{2} \result{}
    and became known as lysine-rich
    whereas the 4 core histones are arginine-rich
    with high arginine/lysine ratios of
    \FPround{\result}{\HTwoAArgLysRatio}{2} \result{} in H2A,
    \FPround{\result}{\HTwoBArgLysRatio}{2} \result{} in H2B,
    \FPround{\result}{\HThreeArgLysRatio}{2} \result{} in H3,
    and \FPround{\result}{\HFourArgLysRatio}{2} \result{} in H4 type isoforms.
    Nevertheless, the core histones contain many lysines particularly in their N-terminal tails.

    Separating histones by polyacrylamide gel electrophoresis (PAGE)
    using the strongly anionic detergent sodium dodecyl sulphate and neutral buffers (SDS PAGE)
    gives single bands for each histone type \citep{ShechterHake2007}.
    However, PAGE with non-ionic detergent Triton X--100 and urea as denaturants
    in acid buffers (TAU or AUT PAGE) allows the separation
    of histone types into multiple bands
    due to post-translational modifications and differences at specific amino acids
    in the polypeptides \citep{Zweidler1977}.
    These TAU PAGE separations gave rise to subtype designations
    H2A.1, H2A.2, H3.1, H3.2, and H3.3.

  \subsection{Functional perspective}

    Canonical core histone expression
    is significantly elevated during S~phase to provide chromatin packaging
    for DNA duplicated during replication \citep{WuBonner1981}.
    This led to their description as ``replication dependent'',
    although a supply of canonical histones is inevitably required
    to partner variants throughout the cell cycle.
    Metazoan canonical core histone genes are distinctive
    because they lack introns and give rise to non-polyadenylated protein coding transcripts.
    Transcript turnover is regulated via a highly conserved 3' stem-loop \citep{MarzluffNatRevGen2008}
    (\tref{tab:typical-histone-differences}).

    In contrast, variant histones such as H2A.Z, TH2B, H3.3, and CENP-A have
    reduced sequence identity and lower abundance \citep{TalbertHenikoff2010}.
    They play functionally specific roles and are mostly expressed outside S~phase,
    so are described as ``replication independent''.
    Since histone variants are interpreted as taking the place
    of equivalent canonical core histone types,
    they are referred to as ``replacement'' histones.

  \subsection{Genomic perspective}

    Canonical core histone genes are found in \NumberOfClusters{} clusters.
    The multiple gene copies in these clustered arrays are
    sometimes confusingly referred to as ``allelic''
    and the resulting combined protein isoforms are often considered to be ``wild type''
    although both genes and protein products display
    variation in primary sequence and relative abundance,
    and their functional equivalence has not been tested.
    In contrast, almost all variant histones are encoded by single genes dispersed in the genome
    with typical properties including introns, alternative splicing,
    and polyadenylated transcripts (\tref{tab:typical-histone-differences}).

  \subsection{Cataloguing of canonical core histone diversity}

    Despite the importance of histones for chromatin organisation and extensive interest
    in their role in epigenetics and regulation, the
    curation and classification of human histone
    gene and protein sequences has not been systematically revisited
    since the landmark survey by \citet{Marzluff02}.
    The Histone Database has for many years provided an online database of
    histone sequences from across eukarya using sequence homology searching
    \citep{HistoneDB1996,HistoneDB1997,HistoneDB1998,HistoneDB1999,
      HistoneDB2000,HistoneDB2002,HistoneDB2004,HistoneDB2006,HistoneDB2011,
      HistoneDB2016},
    while some manually collated listings including human histones have also been undertaken
    \citep{Ederveen2011,HIstome2012,ElkennaniGovin2017}.
    This reflects continual revisions in genome sequence databases
    and the ongoing need for information about human histones.
    In fact, differences in \TotalChangesSinceReference{}~canonical core histone gene details
    (\tref{tab:difference-from-Marzluff02})
    have so far accumulated since the original 2002 survey
    due to rich annotations continuing to propagate into reference sequence databases.

    In this manuscript we provide a comprehensive catalogue
    of canonical core histone genes, encoded proteins, and pseudogenes
    using reference genome annotations as the originating source.
    This reveals a surprising and seldom recognised variation in encoded histone proteins
    that exceeds commonly used subtype designations
    but whose functional implications have not been investigated.

    Since curation and annotation of reference databases are dynamic and evolving, we have
    implemented the manuscript so that it can be regenerated at any time
    from the most current data in the NCBI RefSeq database in order to maintain
    its ongoing value as a reference source in an accessible format.
    All figures and tables were automatically generated using NCBI RefSeq
    data from \printdate{\SequencesDate{}}.
    All dynamically generated values in this copy of the text are displayed
    with a light grey background.
    The manuscript generation process has remained stable in our laboratory for several years
    and represents an example of ``reproducible research'' \citep{Claerbout2000} that
    provides a novel model for cataloguing gene families.
