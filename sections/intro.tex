\section{Introduction}

  Histones are among the most abundant proteins in eukaryotic cells
  and contribute up to half the mass of chromatin.
  The core histone types H2A, H2B, H3, and H4
  define the structure and accessibility of the nucleosome
  as the fundamental repeating unit of genome organisation
  around which the DNA is wrapped \citep{Luger1997structure}.
  In addition, the many chemically reactive sidechains of histones
  are post-translationally modified
  as a nexus for signalling and heritable epigenetics.

  Histones are delineated as either canonical or variant based on
  their gene location, expression characteristics
  and functional roles (\tref{tab:typical-histone-differences}).
  Canonical histones contribute the majority of proteins to
  the bulk structure and generic function of chromatin,
  and are encoded by \TotalGenes{} genes in \NumberOfClusters{}
  clusters in the human genome,
  of which \TotalCodingGenes{} are coding genes and \TotalPseudoGenes{}
  are pseudogenes (\tref{tab:histone-gene-count}).

  \begin{table}
    \caption{Properties distinguishing canonical and variant histone proteins.}
    \label{tab:typical-histone-differences}
    \centering
    \begin{tabular}{l l l}
      \toprule
      \null                     & Canonical             & Variants \\
      \midrule
      Expression timing         & Replication dependent & Replication independent \\
      Sequence identity         & High                  & Low \\
      Functional relationships  & Isoforms              & Specialised functions \\
      Transcript stabilisation  & Stem-loop             & poly(A) tail \\
      Gene distribution         & Clusters              & Scattered \\
      \bottomrule
    \end{tabular}
  \end{table}

  \begin{table}
    \caption{Count of human canonical histone coding and pseudogenes by
             histone cluster and type. $\psi$ indicates pseudo genes.}
    \label{tab:histone-gene-count}
    \centering
    \begin{tabular}{l *{5}{r!{+}r<{$\psi$}}}
      \toprule
      \null   & \multicolumn{2}{c}{H2A}  & \multicolumn{2}{c}{H2B}
              & \multicolumn{2}{c}{H3}   & \multicolumn{2}{c}{H4}
              & \multicolumn{2}{c}{Total} \\
      \midrule
      HIST1   & \HTwoACodingInHISTOne{}     & \HTwoAPseudoInHISTOne{}
              & \HTwoBCodingInHISTOne{}     & \HTwoBPseudoInHISTOne{}
              & \HThreeCodingInHISTOne{}    & \HThreePseudoInHISTOne{}
              & \HFourCodingInHISTOne{}     & \HFourPseudoInHISTOne{}
              & \CodingGenesInHISTOne{}     & \PseudoGenesInHISTOne{} \\
      HIST2   & \HTwoACodingInHISTTwo{}     & \HTwoAPseudoInHISTTwo{}
              & \HTwoBCodingInHISTTwo{}     & \HTwoBPseudoInHISTTwo{}
              & \HThreeCodingInHISTTwo{}    & \HThreePseudoInHISTTwo{}
              & \HFourCodingInHISTTwo{}     & \HFourPseudoInHISTTwo{}
              & \CodingGenesInHISTTwo{}     & \PseudoGenesInHISTTwo{} \\
      HIST3   & \HTwoACodingInHISTThree{}   & \HTwoAPseudoInHISTThree{}
              & \HTwoBCodingInHISTThree{}   & \HTwoBPseudoInHISTThree{}
              & \HThreeCodingInHISTThree{}  & \HThreePseudoInHISTThree{}
              & \HFourCodingInHISTThree{}   & \HFourPseudoInHISTThree{}
              & \CodingGenesInHISTThree{}   & \PseudoGenesInHISTThree{} \\
      HIST4   & \HTwoACodingInHISTFour{}    & \HTwoAPseudoInHISTFour{}
              & \HTwoBCodingInHISTFour{}    & \HTwoBPseudoInHISTFour{}
              & \HThreeCodingInHISTFour{}   & \HThreePseudoInHISTFour{}
              & \HFourCodingInHISTFour{}    & \HFourPseudoInHISTFour{}
              & \CodingGenesInHISTFour{}  & \PseudoGenesInHISTFour{} \\
      \addlinespace
      Total   & \HTwoACodingGenes{}       & \HTwoAPseudoGenes{}
              & \HTwoBCodingGenes{}       & \HTwoBPseudoGenes{}
              & \HThreeCodingGenes{}      & \HThreePseudoGenes{}
              & \HFourCodingGenes{}       & \HFourPseudoGenes{}
              & \TotalCodingGenes{}       & \TotalPseudoGenes{} \\
      \bottomrule
    \end{tabular}
  \end{table}

  Relationships within the histone family have been described using a variety of terminologies
  reflecting biochemical, functional and genomic perspectives that are briefly described below
  and summarised in \tref{tab:histone-divisions}.

  %% The following is a rather long table, longer than 1 page so it can't
  %% go in a float.  There are several packages such as longtable or xtab
  %% but those must be really used with tables.  However what we really
  %% have is a list of descriptions.  If we converted this into a table,
  %% like we had before, it would be a huge waste of space.  So we use
  %% afterpage and captionof to mimic a floating table.  And we use the
  %% shaded environment just because it looks good (although it does
  %% introduce an annoyingly long vertical space before the caption).
  \afterpage{
    \captionof{table}{Terminology describing histone variation}
    \label{tab:histone-divisions}
    \definecolor{shadecolor}{gray}{0.9}
    \begin{shaded}
      \begin{description}
        \item[Allelic variants] \hfill \newline
        Copies of a canonical histone type gene. Not alleles by genetic
        definition. Not histone variants.

        \item[Canonical histones] (complement of variant histones)\hfill \newline
        Core histones as described in \tref{tab:typical-histone-differences}.
        Complement of variant histones in core histones, and contribute to
        the majority of histones in chromatin.  Encompasses multiple
        isoforms.

        \item[Core histones] (complement of linker histones) \hfill \newline
        Histones forming part of the nucleosome core particle wrapping 147 bp of DNA.
        The 4 core histone types are H2A, H2B, H3, and H4.
        Encompasses both canonical and variant histones.

        \item[Heteromorphous variants] \hfill \newline
        Histone variants with distinct function and localisation that are
        readily separated (e.g. SDS PAGE).

        \item[Homomorphous variants] \hfill \newline
        Canonical histone subtypes with isoform variation
        requiring high resolution separation methods (e.g. TAU PAGE).

        \item[Families] \hfill \newline
        Same as ``types''.

        \item[Isoforms] \hfill \newline
        Proteins with high sequence identity and assumed equivalent function.
        Functional equivalence has not been demonstrated for canonical histone isoforms.

        \item[Linker histones] (complement of core histones)\hfill \newline
        Histones binding to linker DNA adjacent to the nucleosome core particle.
        The two linker histone types are H1 and H5.

        \item[Non-allelic variants] \hfill \newline
        Same as ``variant histones''.

        \item[Replacement histones] \hfill \newline
        Same as ``variant histones''.

        \item[Replication-dependent] \hfill \newline
        Same as ``canonical histones''.  So named because their expression
        occurs primarily during S phase.

        \item[Replication-independent] \hfill \newline
        Same as ``variant histones''.  So named because their expression is
        decoupled from the cell cycle.

        \item[Subtypes] \hfill \newline
        Canonical histone type isoforms separable by TAU PAGE
        (e.g. H2A.1, H2A.2).  Despite their naming, there is no evidence of
        functional differences between them.

        \item[Types] \hfill \newline
        Histone sequences able to participate in specific combinations
        that define the repeating nucleosome structure.
        The 5 histone types are H1, H2A, H2B, H3 and H4.

        \item[Variant histones] (complement of canonical histones)\hfill \newline
        Core histones as described in \tref{tab:typical-histone-differences}.
        Complement of canonical histones in core histones.  They perform
        specialized functions and contribute to a minor proportion of
        histones in chromatin.

        \item[Wild type histones] \hfill \newline
        Same as ``canonical histones''.
      \end{description}
    \end{shaded}
  }

  \subsection{Biochemical perspective}

    Abundant histone proteins are readily isolated using their
    highly basic chemical character.
    Successive improvements in fractionation ultimately revealed 5 main histone types
    with nomenclature H1, H2A, H2B, H3, and H4 \citep{nomenclature}.
    An additional H1--related histone H5 is recognised in avian erythrocytes \citep{HFive-review}.

    The demonstration of the nucleosome as the fundamental
    repeating unit of chromatin \citep{Kornberg1974}
    revealed that H2A, H2B, H3, and H4 associate as an octamer of two copies each within the
    nucleosome core particle. These four histones are referred to as core histones.
    In contrast, H1 associates with the linker DNA between nucleosome core particles
    and is referred to as a linker histone.
    The 5 human somatic H1 isoforms and 6 tissue-specific
    variants are described elsewhere \citep{HarshmanFreitas2013}.

    Arginine and lysine content was used as an early distinction between the histones.
    The H1 linker histone has a low arginine/lysine ratio
    of \LinkerArgLysRatio{} and became known as lysine-rich
    whereas the 4 core histones are arginine-rich
    with high arginine/lysine ratios of \HTwoAArgLysRatio{} in H2A, \HTwoBArgLysRatio{} in H2B,
    \HThreeArgLysRatio{} in H3, and \HFourArgLysRatio{} in H4 type isoforms.
    Nevertheless, the core histones contains many lysines particularly in their N-terminal tails.

    Separating histones by polyacrylamide gel electrophoresis (PAGE)
    using the strongly anionic detergent sodium dodecyl sulphate and neutral buffers (SDS PAGE)
    gives single bands for each histone type \citep{ShechterHake2007}.
    However, PAGE with non-ionic detergent Triton X--100 and uncharged urea as denaturants
    in acid buffers (TAU or AUT PAGE) allows the separation
    of histone types into multiple bands
    due to post-translational modifications and differences at specific amino acids
    in the polypeptide \citep{Zweidler1977}.
    These TAU PAGE separations gave rise to subtype designations
    H2A.1, H2A.2, H3.1, H3.2, and H3.3.

  \subsection{Functional perspective}

    Canonical core histone expression
    is significantly elevated during S~phase to provide chromatin packaging
    for DNA duplicated during replication.
    This led to their description as ``replication dependent'',
    although a supply of canonical histones is inevitably required
    to partner variants throughout the cell cycle.
    Metazoan canonical histone genes are distinctive
    because they lack introns and give rise to non-polyadenylated protein coding transcripts.
    Turnover is independently regulated via a highly
    conserved 3' stem-loop (\tref{tab:typical-histone-differences}).

    In contrast, variant histones such as H2A.Z, TH2B, and H3.3 have
    less sequence identity and lower abundance.
    They play functionally specific roles and are mostly expressed outside S~phase,
    so are described as ``replication independent''.
    Since histone variants are interpreted as taking the place
    of equivalent canonical histone types,
    they are also referred to as ``replacement'' histones.

  \subsection{Genomic perspective}

    Canonical histone genes are found in \NumberOfClusters{} clusters,
    possibly for regulatory efficiency.
    The multiple gene copies in these clustered arrays are
    sometimes confusingly referred to as ``allelic''
    and the resulting combined protein isoforms are often considered to be ``wild type''
    although both genes and protein products display
    variation in sequence and relative abundance,
    and their functional equivalence has not been tested.
    In contrast, almost all variant histones are encoded by single genes dispersed in the genome
    with typical properties including introns, alternative splicing,
    and polyadenylated transcripts (\tref{tab:typical-histone-differences}).

  \subsection{Canonical histone diversity}

    Despite the importance of histones for chromatin organisation and extensive interest
    in their role in epigenetics and regulation, the systematic
    curation and classification of human histone
    gene and protein sequences has not been revisited
    since the landmark survey of \citet{Marzluff02}.
    The canonical histone catalogue has accumulated
    \FPeval{result}{round(\SequenceChangeSinceReference{}
                          + \PseudoSinceReference{}
                          + \CodingSinceReference{}
                          + \AddedSinceReference{}
                          + \RemovedSinceReference{}, 0)}
    \result{}~differences (\tref{tab:difference-from-Marzluff02})
    from that survey as rich annotations continue to propagate
    into reference sequence databases.

    In this manuscript we provide a comprehensive catalogue
    of canonical histone genes, proteins, and pseudogenes.
    Since curation and annotation are dynamic and evolving,
    we have implemented the manuscript so that it can
    self-update from the most current data in the NCBI RefSeq database
    in order to maintain its value as a reference source in an accessible format.
    All automatically generated values here are displayed with a light grey background
    and were calculated using NCBI RefSeq data on \SequencesDate{}.
    \todo{Can't we generate linux timestamp and apply a formatting string in latex?}.
    This manuscript generation process has remained stable in our laboratory for several years
    and represents an example of ``reproducible research'' \citep{Claerbout2000}
    that provides a novel model for curating relatively stable gene families.
