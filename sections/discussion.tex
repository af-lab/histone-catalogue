\section{Canonical isoforms vs.~variants}
  %% Characteristics of a variant are the fact they seem to have special functions. Emphasis
  %% that some canonical may have special functions which no one bothered to check
  %% variants with extra special nomenclature are outside the histone clusters and have poly-A tails (sometimes)
  %% Complain about the lack of information on the different isoforms
  \todo[inline]{Rules and exceptions to definition of canonical and variants. Also point
              that the word canonical suggest antecedence which is not necessarily true (ref to our
              paper and Henikoff's)}

  Variants are the most likely with a standard gene, are considered independent from each other,
  spread around the genome and its expression is mostly independent of cell cycle \todo{explain
  why mostly? Do it next paragraph when complaining the naming}. When people say H2AX that's
  a variant. Note, H2A.2 and H3.2 are not considered variants.

  Both this separation and their naming is very misleading \todo{say that the fact that no one cares is proof?}
  for several reasons.
  \begin{itemize}
    \item The word canonical suggests precedence which is not necessarily true. It has been
          suggested that at least in the case of H2A, the `canonical' gene may have evolved
          from one of the `variant' genes\addref.
    \item Not all `variant' are replication-independent. H2AX gene for example expression
          increases during S--phase together with other `canonical' histones and does show
          a transcript with stem-loop.
    \item The `variant' vs `canonical' separation suggests that there's no variance inside
          the `canonical' which is false. There's several unique protein sequences for each
          `canonical' protein. And even the sometime named H2A.2 and H3.2 variants are actually
          `canonical' histone proteins.
  \end{itemize}

