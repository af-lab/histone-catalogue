\section{Canonical isoforms vs.~variants}
  %% Characteristics of a variant are the fact they seem to have special functions. Emphasis
  %% that some canonical may have special functions which no one bothered to check
  %% variants with extra special nomenclature are outside the histone clusters and have poly-A tails (sometimes)
  %% Complain about the lack of information on the different isoforms

  %% Should we be pointing the finger to bad examples? That's really bad form.
  %% We could show examples where H2AX is called histone isoform, or even
  %% canonical histones are named variants, but really, what's the pointing of
  %% pointing the fingers to those who are doing it wrong other than shame them?

  The terms canonical, variant, and isoform histones are often misused, although
  the correct meaning can usually be infered from the context. This
  confusion is likely to stem from not knowing the existence of the two groups,
  rather than confusion about where they belong.

  There are canonical and variant histones, each group listed in \tref{tab:histone-catalogue}
  and~\ref{tab:variant-catalogue} respectively. Within these two groups,
  there are also isoforms. For example, each of the \HTwoAUniqueProteins{} unique H2A proteins
  is an isoform of the canonical H2A, while the multiple H2AFY are isoforms of
  the H2AFY protein.

  We have summarized the distinction between canonical and variants in
  \tref{tab:typical-histone-differences} which is then reflected on their
  gene symbols (\fref{fig:nomenclature}). While this separation appears like a
  good organization, findings of new genes and studies start to stress those
  definitions to the point where their distinction is unclear. And since
  gene symbols should be stable, such distinction is actually harmful. We will
  either end up with changing or meaningless symbols.

  %% "Attack" each of the identifying characteristics

  %% The paper describing H2AFJ, with the RT-PCR has a plotthat shows H2AFJ
  %% levels stable while HIST2H2AA2 increases during S phase. But shouldn't
  %% canonical histones increase 35 times instead of only 2.3? On the other
  %% hand, the plot does not have legend for either the X or Y axis, I can
  %% probably be interpretting it wrong.

  The main distinction between the two groups is related to its dependence on
  cell cycle. Canonical histones are mainly expressed during S~phase due to
  its mRNA 3' processing via the stem-loop mechanism. This has been used
  as the single deciding characteristic used to name H2AFJ as an histone variant,
  despite no knowledge of specialized function\citep{h2afj-report}. In the study,
  it is shown that H2AFJ has a poly-A tail instead and not a stem-loop, and that
  its expression levels do not increase during S--phase.

  However, it is now also argueed that some canonical H2B histones genes, encode
  two transcripts, with a stem-loop and with a poly(A) tail. This suggest
  that not all of the canonical histones are completely replication dependent.
  An equivalent system is used by the variant H2AX which is usually regarded
  as replication independent.

  %% TODO the sanity check list at the end of the paper, should be made into a
  %%      a table, or something else that can be referenced, so we can use it
  %%      in the text

  In addition, the variant vs.~canonical separation suggests that there's no variance inside
  the canonicals which is false. There's several unique protein sequences for each
  `canonical' protein. And even the sometimes named H2A.2 and H3.2 variants are actually
  canonical histone proteins, with no more sequence variance than the other canonical
  isoforms.

  Finally, the word ``canonical'' is suggests precedence, the original histones
  from which variants diverged. We have suggested, as others did, that this
  is not necessarily true, and that it may have been the H2AX ``variant'' the
  original from which the ``canonical'' H2A evolved\citep{our-H2AX-review, henikoff2010-variants-evolution}.


  Variants are the most likely with a standard gene, are considered independent from each other,
  spread around the genome and its expression is mostly independent of cell cycle \todo{explain
  why mostly? Do it next paragraph when complaining the naming}. When people say H2AX that's
  a variant. Note, H2A.2 and H3.2 are not considered variants.


  However, the term ``variant histones'' implies that canonical histones are
  homogeneous and masks their heterogenity since for sure, if another histone
  of the same type is different, it ought to be a variant. This is not
  true. Each variant, such as H2A.Z, H2Abbd, and H3.3,
  is encoded by a gene, with its own sequence and name,
  but canonicals are encoded by multiple genes with less varying sequence.

  It is often said in software development, that no documentation is better
  than incorrect documentation. We stand by this opinion and arguee that
  a nomenclature that is sometimes incorrect does more wrong than right.
  Incorrect documentation of software
  is usually the result of evolving code whose documentation is left behind,
  not of documentation incorrectly written at the start.
  This is the same with histone gene nomenclature. It correctly delivered
  the, now outdated, information about them. However, as the knowledge of
  these genes increased,
  their symbols simply become misleading and a source of wrong assumptions.

  A list of all the changes since \citet{Marzluff02} is
  displayed on \tref{tab:difference-from-Marzluff02}.


  With this article being closer to a software release than a paper publication,
  patches are welcome.


  when the gene
  symbol is given there is no doubt ever.

